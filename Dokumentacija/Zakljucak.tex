\chapter{Zaključak i budući rad}
			
		Izrada aplikacije je započeta analiziranjem zadanih uvjeta te pretakanjem korisničkih zahtjeva u formalne zapise uvjeta. Prilikom popisivanja zahtjeva dodani su zahtjevi koje smo smatrali da bi bili od koristi korisnicima. Kasnije se pokazalo da to nije bio dobar potez. \newline
		Nakon formaliziranja zahtjeva korisnika izrađen je kostur aplikacije te su isprobane jednostavne funkcije.\newline
		Ovime je završila prva faza projekta. Dobar potez je bio što smo simultano radili na aplikaciji i usklađivali zahtjeve u dokumentaciji. Loš potez je bio što smo pri izradi kostura krenuli s izradom CSS-a za stranicu što sada smatramo da je trebalo napraviti na kraju. Problem nam je predstavljalo nepoznavanje tehnologije git zbog čega smo puno vremena gubili na raznim konfliktima. Prvi put smo se susreli s Google maps API-jem te je bilo potrebno neko vrijeme da se naviknemo na rad s njim; sada se osjećamo ugodno raditi s njim.
		
		U drugoj fazi projekta smo dovršili aplikaciju i kod. Puno smo se sigurnije osjećali eksperimentirati s tehnologijama pošto smo stekli iskustvo. Ovo je vrlo vjerojatno iz razloga što nismo odlagali izradu za pred rok puštanja u pogon već smo radili kontinuirani pa smo imali vremena i popraviti ako nešto krivo napravimo.\newline
		
		Kako je kod aplikacije postao opsežniji tako smo počeli imati sve više konflikata koje smo rješavali tako što smo prestali u isto vrijeme raditi na istim datotekama. Razlog velikog broja konflikata i gubljenja vremena je što su se često implementirale funkcionalnosti koje su loše dokumentirane i jer nismo slijedili dobre prakse programiranja. Smatrali smo kako naše male promijene u kodu su zanemarive te da ih nije potrebno strukturirati. Iz ovoga smo zaključili kako je bitno od početka dobro strukturirati i dokumentirati kod. Najveća korist koju smo izvukli iz ovog projekta je praksa u zajedničkom radu. Ovo je prvi put da smo morali napraviti zajedno projekt srednje veličine te smatramo to dragocjenim iskustvom. Također smo naučili raditi s novim tehnologijama te podsjetili se nekih stari i produbili svoje znanje o njima.\newline
		
		Zaključili smo kako postoji puno nejednoznačnosti u zahtjevima naručitelja te smo naučili važnost ispravnog i pravovremenog komuniciranja. Valja istaknuti kako je od velike koristi bilo od početka razrješiti što više nedoumica jer kako smo napredovali s izradom aplikacije to je bilo teže prilagoditi kod i dokumentaciju željama naručitelja aplikacije.\newline
		
		Cijeli proces bi bio puno brži i ugodniji da smo već radili kao tim te da je svaki član tima imao specijalno znanje u određenoj tehnologiji, a ne da svaki član radi malo u svakoj tehnologiji. Unatoč neekspertnosti smo se s vremenom podijelili na različite tehnologije te time znatno povećali produktivnost i smanjili konflikte.\newline
		
		Na kraju smo zaključili da je bitnije napraviti što bolje zahtjeve naručitelje nego pokušati dodati nove funkcionalnosti ali izgubiti na kvaliteti koda. Sve funkcionalnosti koje smo mislili implementirati smo implementirali, iznimka su one funkcionalnosti koje smo mislili na početku da ih ima smisla raditi ali se ispostavilo da nisu potrebne (na primjer postojao je use case za brisanje automobila, dodavanje automobila i za uređivanje automobila, posljednji je beskoristan jer se njegova funkcionalnost može riješiti s prva dva).\newline
		
		Zahvalni smo profesorima koji su nas usmjeravali prema pravom putu pri izradi aplikacije i odgovarali na sva pitanja koja smo imali.\newline
		
		Voljeli bi jednoga dana nastaviti raditi na ovom projektu ili sličnom.\newline
		
		NDB
		
		\eject 